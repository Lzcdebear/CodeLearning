\documentclass{ctexart}
\usepackage{amsmath}
\usepackage{float}
\usepackage{amssymb}
\usepackage{graphicx}
\usepackage{gbt7714}
\usepackage{pifont}
\usepackage{wrapfig}
\usepackage{multirow}
\usepackage{array}

\ctexset{
    % 修改 section。
    section={   
        name={,、},
        number={\chinese{section}}
    }
}

\title{磁滞回线}
\author{陆知辰-10225301478}
\date{\today}
\graphicspath{{figure/}}

\begin{document}

\begin{titlepage}
  \centering
  % 插入图片
  \includegraphics[width=0.5\textwidth]{ecnu.png}
  
  % 空行用于调整标题位置
  \vspace*{\baselineskip}
  
  % 标题
  \Huge\textbf{物\quad 理\quad 实\quad 验 \quad (二)}
  % 空行用于调整标题和其他信息之间的间距
  \vspace*{0.3\baselineskip}
  
  % 具体实验名称
  \huge 磁滞回线
  
  % 空行用于调整时间和其他信息之间的间距
  \vspace*{2\baselineskip}
  
  % 时间
  \large 时间:\today
  
  % 空行用于调整时间和其他信息之间的间距
  \vspace*{\baselineskip}
  
  % 创作人
  \large 创作人:陆知辰
  
  % 空行用于调整创作人和学号之间的间距
  \vspace*{\baselineskip}
  
  % 学号
  \large 学号:10225301478
  
\end{titlepage}
\newpage
\tableofcontents
\newpage
\section{实验摘要}
  \subsection{实验概要}
  铁磁物质是一种性能特异、用途广泛的材料.铁、钴、镍及其众多合金以及含铁的氧化物均属铁磁物质,
  铁磁物质的一个特征是在外磁场作用下能被强烈磁化,故铁磁物质的磁导率很高,铁磁物质的另一个特征是磁滞现象,即磁化场停止作用后,
  铁磁物质仍会保留磁化状态,磁滞现象有着广泛的应用.

  \subsection{实验目的}
  1.\quad 认识铁磁物质的磁化规律,比较不同铁磁材料的动态磁化特性。

  2.\quad 了解利用示波器测量铁磁材料动态磁滞回线的原理和方法。
  
  3.\quad 测绘铁磁样品的磁滞回线和基本磁化曲线。

\section{实验原理}
  \subsection{铁磁材料的磁滞现象}
  图\ref{tieciqvxian}为铁磁物质磁感应强度B与磁场强度H之间的关系曲线。
  图中的原点O表示磁化之前铁磁物质处于磁中性状态,即B=0,H=0.当磁场的H从零开始增加时,磁感应强度B随之缓慢上升,
  如线段Oa所示;继之B随H迅速增长,如线段ab所示;其后B的增长又趋缓慢,并当H增至Hm时,
  B到达饱和值.OabS称为起始磁化曲线。
  当磁场从Hm逐渐减小至零,磁感应强度B并不沿起始磁化曲线恢复到原点O,
  而是沿另一条新曲线SR下降。
  比较线段OS和SR可知,减小B相应也减小,但B的变化滞后于H的变化,这种现象称为磁滞,磁滞的明显特征是当H=0时,B不为零,而保留剩磁B。

  \begin{figure}[H]\label{tieciqvxian}
    \centering
    \includegraphics[width=0.8\textwidth,height=0.4\textheight]{tieciqvxian.jpg}
    \caption{铁磁材料磁滞回线}
  \end{figure}

  当磁场反向从O逐渐变至$-H_{e}$。时,磁感应强度B消失.这说明要消除剩磁,
  必须施加反向磁场,$H_{e}$称为矫顽力,它的大小反映铁磁材料保持剩磁状态的能力,线段RD称为退磁曲线。

  图\ref{tieciqvxian}还表明,当磁场按$$H_{m}\rightarrow 0\rightarrow -H_{e}\rightarrow -H_{m}\rightarrow 0\rightarrow H_{e}\rightarrow H_{m}$$
  次序变化,相应的磁感应强度B则沿闭合曲线SRDS'R'D'S'变化,这条闭合曲线称为磁滞回线。
  所以,当铁磁材料处于交变磁场中时(如变压器中的铁芯),将沿磁滞回线反复被磁化→去磁→反向磁化→反向去磁,
  在此过程中要消耗额外的能量,并以热的形式从铁磁材料中释放,这种损耗称为磁滞损耗,可以证明,磁滞损耗与磁滞回线所围面积成正比。

  应该说明,当初始态为H=0,B=0的铁磁材料,在交变磁场强度由弱到强依次进行磁化,可以得到面积由小到大向外扩张的一簇磁滞回线,
  如图\ref{tongyicailiaoqvxian}所示.这些磁滞回线顶点的连线称为铁磁材料的基本磁化曲线,由此可近似确定其磁导率$\mu =\frac{B}{H}$。
  因B与H的关系成非线性,故铁磁材料的$\mu$不是常数,
  而是随H而变化,如图\ref{muH}所示.铁磁材料的相对磁导率可高达数千乃至数万,这一特点是它用途广泛的主要原因之一。

  \begin{figure}[H]
    \begin{minipage}[t]{0.45\textwidth}
      \centering
      \includegraphics[width=\textwidth,height=0.3\textheight]{tongyicailiaoqvxian.jpg}
      \caption{同一材料的一簇磁滞曲线}
      \label{tongyicailiaoqvxian}
    \end{minipage}
    \begin{minipage}[t]{0.45\textwidth}
      \centering
      \includegraphics[width=\textwidth,height=0.3\textheight]{muH.jpg}
      \caption{铁磁材料$\mu$和$H$的关系}
      \label{muH}
    \end{minipage}
  \end{figure}

  磁化曲线和磁滞回线是铁磁材料分类和选用的重要依据,图\ref{butongcailiaoqvxian}为常见的两种典型的磁滞回线。
  其中软磁材料磁滞回线狭长,矫顽力、剩磁和磁滞损耗均较小,是制造变压器、电机、和交流磁铁的主要材料;而便磁材料磁滞回线较宽,矫顽力大,剩磁强,可用来制造永磁体。

  \begin{figure}[H]\label{butongcailiaoqvxian}
    \centering
    \includegraphics[width=0.5\textwidth,height=0.3\textheight]{butongcailiaoqvxian.jpg}
    \caption{不同材料的磁滞回线}
  \end{figure}


  \subsection{用示波器观察和测量磁滞回线的实验原理和线路}
  在用示波器观察时,示波器工作在XY工作模式,其中x轴输人为磁场强度H,y轴输入为磁感应强度B。观察和测量磁滞回线和基本磁化曲线的线路如图\ref{shiyanyuanliluxian}所示。

  \begin{figure}[H]\label{shiyanyuanliluxian}
    \centering
    \includegraphics[width=0.9\textwidth,height=0.3\textheight]{shiyanyuanxianlu.jpg}
    \caption{实验原理线路}
  \end{figure}

  在图\ref{shiyanyuanliluxian}中待测样品为EI型砂钢片,被制为闭合的环形,然后均匀绕以励磁线圈$N_{1}$,和测量线圈$N_{2}$。
  220V的交流电经电压变换后经过多挡开关$S_{1}$,加到励磁绕组$N_{1}$上,$S_{1}$可以调节加到$N_{1}$,上的电压值。$R_{1}$为可调电阻,
  用来对励磁电流取样,其总值为5$\Omega$。调节$R_{1}$处的多挡开关,即以$0.5\Omega$等间隔改变可调电阻$R_{1}$与示波器并联部分的电阻,
  从而可以调节输入到示波器上的电压$U_{H}$。

  设通过$N_{1}$的交流励磁电流为i,根据安培环路定律,样品的磁场强度为
  
  \begin{equation}
    H=\frac{N_{1} \cdot i}{L}
  \end{equation}

  式中L为样品的平均磁路长度,设$R_{1}$与示波器并联部分的电阻为$R_{1s}$,则有

  \begin{equation}
    U_{H}=R_{1s}i
  \end{equation}

  在交流励磁电路恒定的情况下,$U_{H}$随可调电阻$R_{1}$的$R_{1s}$部分变大而变大。通过$U_{H}$和$R_{1s}$的值可以得到励磁电流的值。有

  \begin{equation}
    H=\frac{N_{1}}{LR_{1s}} \cdot U_{H}
  \end{equation}

  当式中$N_{i}$、$L$、$R_{1}$均已知的常数时,可由$U_{H}$确定$H$的值。

  在交变磁场下,样品的磁感应强度瞬时值B由测量线圈和$R_{2}C_{2}$电路来确定,
  根据法拉第电磁感应定律,由于样品中磁通量$\Phi$的变化,在测量线圈中产生的感生电动势的大小为
  
  \begin{equation}
    \varepsilon_{2} = N_{2} \frac{d\Phi}{dt}
  \end{equation}

  \begin{equation}
    \Phi = \frac{1}{N_{2}} \int \varepsilon_{2}dt 
  \end{equation}

  \begin{equation}
    B=\frac{\varepsilon}{S}=\frac{1}{N_{2}S} \int \varepsilon dt
  \end{equation}

  式中S为样品的截面积。

  如果忽略自感电动势和电路损耗,则回路方程为
  
  \begin{equation}
    \varepsilon_{2} = i_{2} R_{2} + U_{B}
  \end{equation}

  式中$i_{2}$为感生电流,$U_{B}$为积分电容$C_{2}$两端电压,设在$\Delta t$时间内$i_{2}$向电容$C_{2}$
  充电电荷量为Q,则有

  \begin{equation}
    U_{B}=\frac{Q}{C_{2}}
  \end{equation}

  \begin{equation}
    \varepsilon_{2} = i_{2}R_{2} + \frac{Q}{C_{2}}
  \end{equation}

  如果选取足够大的$R_{2}$和$C_{2}$,使$i_{2}R_{2}>>\frac{Q}{c_{2}}$

  \begin{equation}
    \varepsilon_{2}=i_{2}R_{2}=\frac{dQ}{dt}R_{2}=C_{2}\frac{dU_{B}}{dt}R_{2}
  \end{equation}

  上式中可由$U_{B}$确定$B$,则有

  \begin{equation}
    B=\frac{C_{2}R_{2}}{N_{2}S} U_{B}
  \end{equation}

  综上所述,只要将$U_{H}$和$U_{B}$分别加到示波器的“X输入”和“Y输人”便可观察样品的B-H曲线,并可用示波器测出$U_{H}$和$U_{B}$值,进而根据公式计算出B和H。

\section{实验装置器材介绍}
磁滞回线实验仪、示波器

\section{实验内容及实验步骤}
  \subsection{电路连接}
  选择硅钢片材料(蓝色)磁芯,按电路图连接线路,并令$R_{1}=5\Omega$,“U选择”置于0位。$U_{H}$和$U_{B}$,分别接示波器的“X输入”和“Y输入”。

  \subsection{样品退磁}
  开启实验仪电源,转动“U选择”旋钮,令U从0增至3V,然后再转动旋钮,将U从最大值降为0,从而消除剩磁,确保样品处于磁中性状态,即B=H=0,如图\ref{tuicishiyi}所示.

  \begin{figure}[H]
    \begin{minipage}[t]{0.45\textwidth}
      \centering
      \includegraphics[width=\textwidth,height=0.3\textheight]{tuicishiyi.jpg}
      \caption{退磁示意图}
      \label{tuicishiyi}
    \end{minipage}
    \begin{minipage}[t]{0.45\textwidth}
      \centering
      \includegraphics[width=\textwidth,height=0.3\textheight]{jibianxianxiang.jpg}
      \caption{调节不当引起的畸变现象}
      \label{jibianxianxiang}
    \end{minipage}
  \end{figure}

  \subsection{观察退磁回线}
  令U=3.0V,开启示波器电源,并分别调节示波器X和Y轴的灵敏度,使显示屏上出现图形大小合适的磁滞回线,
  若图形顶部出现编织状的小环,如图\ref{jibianxianxiang}所示,这时应该检查示波器的通道输入方式,
  一般应选择“DC”,或者X通道“AC”,Y通道“DC”,并适当选择$R_{1}$值,或降低励磁电压U予以消除。

  \subsection{观察基本磁化曲线}
  按步骤2对样品进行退磁,从U=0开始,
  逐挡提高励磁电压,将在显示屏上得到面积由小到大一个套一个的一簇磁滞回线,
  记录下这些磁滞回线顶点的B和H的值,并将B和H的值作图连线就是样品的基本磁化曲线。

  \subsection{已知条件}
  调节U=3.0 V,$R_{1}=5\Omega$,测定样品的一组$U_{B}$、$U_{H}$值,并根据已知条件:L= 75 mm,S=120 $mm^{2}$,
  $C_{2}=20\mu F$,$R_{2}=10k\Omega$,N=60匝,计算出相应的B和H的值。


  \subsection{$W_{BH}$}
  根据得到的B和H的值作B-H曲线,根据曲线求得$B_{m}B_{r}H_{e}$等参数,并估算曲线的面积来求得$W_{BH}$。

  \subsection{$\mu - H$曲线}
  依次测定U=0.5、1.0、\dots、3.5V十组$U_{H}U_{B}$,计算得出相应的$H_{m}H_{m}\mu$,作出$\mu - H$曲线。


  \subsection{不同曲线观察}
  改变$R_{1}$,观察不同的磁化曲线。

  \subsection{二次测量}
  更换样品为另一个磁芯($N_{1}=90$匝),重复上述步骤,并对比两种材料的测量结果。
\newpage

\section{实验原始数据}
\begin{figure}[H]
  \centering
  \includegraphics[width=0.9\textwidth,height=0.8\textheight]{yuanshishujv1.jpg}
  \caption{实验原始数据1}
\end{figure}
\newpage

\begin{figure}[H]
  \centering
  \includegraphics[width=0.9\textwidth,height=0.8\textheight]{yuanshishujv2.jpg}
  \caption{实验原始数据2}
\end{figure}
\newpage


\section{实验数据处理}
实验数据以及作图如图\ref{cizhihuixianzuotu}所示。

\begin{figure}[H]
  \begin{minipage}[t]{0.45\textwidth}
    \centering
    \includegraphics[width=0.3\textwidth,height=0.8\textheight]{hongseqvxianshujv.png}
    \caption{红色样品磁滞回线}
    \label{hongseshujv}
  \end{minipage}
  \begin{minipage}[t]{0.45\textwidth}
    \centering
    \includegraphics[width=0.3\textwidth,height=0.8\textheight]{heiseqvxianshujv.png}
    \caption{黑色样品磁滞回线}
    \label{heiseshujv}
  \end{minipage}
\end{figure}
\newpage

\begin{figure}[H]
  \centering \label{cizhihuixianzuotu}
  \includegraphics[width=0.9\textwidth,height=0.5\textheight]{zuotu.png}
  \caption{磁滞回线图}
\end{figure}

\section{思考题}
  \subsection{思考题一}
  在测绘磁滞回线之前需要进行退磁的主要原因是消除或减小磁性材料中的残余磁场,确保测量的准确性。如果材料中存在残余磁场,它会对测绘磁滞回线的结果产生干扰,使得测量数据不准确。

  在磁性材料上施加一个逐渐减小的直流磁场,然后逐渐减小到零。这个过程中,磁性材料的磁矩会随着外加磁场的变化而重新排列,最终达到退磁的效果。

  \subsection{思考题二}
  软磁性的磁滞回线的面积比较小,而硬磁性的磁滞回线的面积比较大。

  \subsection{思考题三}
  通过电压的值来代表H和B之间的值。通过电磁感应原理进行实验。

  \subsection{思考题四}
  调节不同的情况下的磁滞回线的样子,并便于研究磁化曲线。

\section{实验中个人的思考与感想}
  \subsection{对于实验个人观点}
  实验测量可以使用示波器直接进行描点,也可以通过curse进行描点。前者的误差比较大,而且比较难以描绘。

  实验得到的两个样品的磁滞回线的确不同,但是和书上的图片的差距还是比较大的。可能由于实验中两个材料的性质决定的。

  \subsection{实验中的总结}
  实验测量不同样品的磁滞回线,通过测量电压值进行测量。能够观察到软磁性和硬磁性的铁磁材料的不同图像。
\end{document}
