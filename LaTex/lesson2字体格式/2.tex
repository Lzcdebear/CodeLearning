% 导言区
\documentclass{article}

\usepackage{ctex}
\newcommand{\myfont}{\textit{\textbf{\textsf{Fancy Text}}}}
% 正文区
\begin{document}
    % 其中可以使用 {} 来限制字体文本的格式限制文本
    % 设置字体族 罗马字体,打字机字体,无衬线字体 
    \textrm{Roman Family}
    \textsf{Sans Serif Family}
    \texttt{Typewriter Family}

    \rmfamily Roman Family
    \sffamily Sans Serif Family
    \ttfamily Typewriter Family

    % 设置字体系列 粗细 宽度
    \textmd{Medium Series}
    \textbf{Boldface Series}

    {\mdseries Medium Series}
    {\bfseries Boldface Series}

    % 设置字体形状 直立 斜体 伪斜体 小型大写
    \textup{Upright Shape}
    \textit{Italic Shape}
    \textsl{Slanted Shape}
    \textsc{Small Caps Shape}

    {\upshape Upright Shape}
    {\itshape Italic Shape}
    {\slshape Slanted Shape}
    {\scshape Small Caps Shape}

    % 设置字体格式
    % 中文格式
    {\songti 宋体}\quad
    {\heiti 黑体}\quad
    {\fangsong 仿宋}\quad
    {\kaishu 楷书}
    % 中文中的粗体使用的是黑体,而斜体使用的是楷书
    
    % 字体大小
    % 使用的比较大小标准是根据开头的 documentclass 确定的,可以在{}前加入[数字+pt]来设置标准大小的磅数
    {\tiny          Hello}\\
    {\scriptsize    Hello}\\
    {\footnotesize  Hello}\\
    {\small         Hello}\\
    {\normalsize    Hello}\\
    {\large         Hello}\\
    {\Large         Hello}\\
    {\LARGE         Hello}\\
    {\huge          Hello}\\
    {\Huge          Hello}\\
    % 中文字号的使用可以在 cmd 中输入 texdoc ctex 来进行查看
    
    在正文中不建议使用大量的格式,而是应该在导言区内定义一个新命令然后进行修改
    
    比如接下来我将使用myfont进行编译
    \myfont
\end{document}