\documentclass{ctexart}
\usepackage{amsmath}
\usepackage{amssymb}
\bibliographystyle{plain}
% 正文区
% 参考文献区域
\begin{document}
%     参考文献使用方式是在集合的thebibliography环境中进行设置\\
% 设置的时候bibitem后面是对象的名字,也就是之后引用的时候所用的名字\\
% 而之后的内容就是标准的格式对象的名称什么的,在 emph后面包括的内容是重点,会进行斜体或者改变字体的方式进行强调

%     以下就是对于参考文献在正文中的引用的例子

%     对于引用对象可以使用cite进行对象设置,比如引用第一个文献 \cite{article1}

%     对于其他的不止一次的引用的情况,也就是在不同的文件中都需要引用的时候可以使用另一个文件进行
%     参考文献的整理然后再使用,具体参考文献文件参考同一文件夹下的11ref文件
%     \begin{thebibliography}{99}
%         \bibitem{article1}陈立辉,苏伟,蔡川,陈晓云.\emph{基于LaTex的Web数学公式提取方法研究}[J].计算机科学。2014(06)
%         \bibitem{book1}William H. Press,Saul A. Teukolsky,
%         William T. Vetterling,Brian P. Elannery,
%         \emph{Numerical Recipes 3rd Edition:The Art of Scientific Computing}
%         Cambridge University Press, New York,2007.
%         \bibitem{latexGuide} Kopka Helmut, W. Daly Patrick,
%         \emph{Guide to LaTeX}, $4^{th}$ Edition.
%         Available at \texttt{http://www.amazon.com}.
%         \bibitem{latexMath} Graetzer George, \emph{Math Into \LaTeX},
%         BirkhÃxuser Boston; 3 edition (June 22, 2000).
%     \end{thebibliography}
    同样还可以
    use \cite{mittelbach2004}
    \bibliography{ref}
\end{document}
