\documentclass{article}

\usepackage{amsmath}
\usepackage{ctex}
\usepackage{amssymb}

% 正文区
\begin{document}
    %可以使用gather环境进行多行公式的排版
    %gather中如果没有*的话会对公式进行自动编号,使用\\进行换行
    \begin{gather}
        a + b = b + a\\
        av  va
    \end{gather}
    \begin{gather*}
        a + b = b + a\\
        a\times v=v\times a
    \end{gather*}
    \begin{gather}
        a + b = b + a \notag \\
        av  va \notag \\
        life is simple
    \end{gather}
    %在结束的地方加上notag将不会编号
    %align能够帮助进行公式的对齐,对齐的位置是&所定位的地方
    \begin{align}
        aa &= b\\
        b  &= c
    \end{align}
    其他的一些操作可以使用
    equation环境
    进行调整

    比如在equation后split就能完成公式换行对齐,对齐位置依旧是\&

    也可在equation后case就能完成类似分段函数排版的形式,
    在\& 位置处分节,每行都是如此,分节后的下一节自动对齐

    使用equation的好处是就算公式再长由于使用的只有一个equation所以编号
    只会编一次
\end{document}
