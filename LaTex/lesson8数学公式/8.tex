\documentclass{ctexart}
\usepackage{amsmath}

% 正文区
\begin{document}
    \section{简介}
    LaTeX 将排版内容分为文本模式和数学模式,
    文本模式用于普通的文本排版,
    数学模式用于数学公式排版
    \section{行内公式}
    \subsection{美元符号}
    交换律是 $a+b=b+a$,如$1+2=2+1$
    \subsection{小括号}
    交换律是 \(a+b=b+a\),如\(1+2=2+1\)
    \subsection{math环境}
    交换律是 
    \begin{math}
        a+b=b+a
    \end{math}

    \section{上下标}
    \subsection{上标}
    $3x^2 - x + 2 = 0$
    $3x^{20} - x + 2 = 0$
    \subsection{下标}
    $a_0,a_1,a_2$
    $a_{100}$
    \section{希腊字母}
    $\alpha$
    $\beta$
    $\gamma$
    $\epsilon$
    $\pi$
    $\omega$

    $\Delta$
    $\Theta$
    $\Omega$

    $\alpha^2 + \beta^{2\omega-3} = \epsilon^{\delta}$
    \section{数学函数}
    $\log$
    $\sin$
    $\cos$
    $arcsin$
    $ln$

    $\sqrt[7]{x^2}$
    \section{分式}
    大约是原体积的$3/4$
    大约是原体积的$\frac{3}{44}$
    \section{行间公式}
    \subsection{美元符号}
    可以将公式放在两个美元符号构成的内容中,行间公式将会自动居中然后进行编号

    交换律是$$a+b=b+a$$例如$$1+2=2+1$$
    \subsection{中括号}
    交换律是\[a+b=b+a\]例如\[1+2=2+1\]
    \subsection{displaymath环境}
    交换律是
    \begin{displaymath}
        a+b=b+a
    \end{displaymath}
    for example:
    \begin{displaymath}
        1+2=2+1
    \end{displaymath}
    \subsection{自动编号的equation环境}
    对于equation环境下的公式可以进行标签然后进行引用

    交换律的公式就是式\ref{eq:commutative}
    \begin{equation}
        \label{eq:commutative}
        a+b=b+a
    \end{equation}
    \subsection{不自动编号的equation*环境}
    引用公式\ref{eq:commutative2}
    \begin{equation*}\label{eq:commutative2}
        a+b=b+a
    \end{equation*}
    如果对equation*进行引用的话默认的编号是小节的编号比如

    这一段的小节是 7.5,那进行引用的时候引用的编号就是7.5
\end{document}