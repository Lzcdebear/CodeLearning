% 导言区
\documentclass{article} % book report letter
% 这里还有一个方法就是直接将文件格式从 article 改为 ctex 的
% 其中包括 ctexart ctexrep ctexbook,对应原版,除了 letter
\usepackage{ctex}
% 如果想要在latex中使用中文需要以下操作
% 将编译器改为 xelatex
% 将编码格式改为 utf-8

% 如果出现latex中没有的函数可以使用定义的方式
% 以下就是定义一个符号叫做 \degree 目的就是右上显示度的符号
\newcommand\degree{^\circ}
\title{\heiti 杂谈为什么物理这么难}
\author{Nicolai}
\date{\today}


% 正文区
\begin{document}
    \maketitle    
    % 其中的book report letter
    % 百分号就是相似 python 中的 井号
    % $$符号中间的部分会以公式格式输出,符号外面部分则为文本
    % 如果想要输出的文本换行则需要空一行
    % $$ $$之间文本同样为数学模式下的输出,但是会居中换行
    Hello World!
    let $f(x)$ be defined by $f(x) = x + y$

    Hello World!
    % 如果空行内有注释将不会换行,必须要再空一行

    let $f(x)$ be defined by $$f(x) = x + y$$ this formula is writen by myself
    
    % 如果使用之前的设置,那在正文中将可以使用中文
    加油,今天又是被生活战胜的一天
    
    并且可以在文本中插入一些公式就比如

    勾股定理用数学的方式表达就是在三角形$ABC$中,假设$\angle C=90\degree$,则会有
    \begin{equation}
        AB^2 = BC^2 + AC^2
    \end{equation}
    % 这里的equation将会生产和 $$ 一样的效果的公式,但是在最右侧将会有编号
\end{document}