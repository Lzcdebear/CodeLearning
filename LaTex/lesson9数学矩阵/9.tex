\documentclass{article}

\usepackage{amsmath}
\usepackage{ctex}

% 正文区
\begin{document}
    \[
        % matrix格式,在矩阵两侧没有括号
        \begin{matrix}
            0 & 1 \\
            1 & 0
        \end{matrix} \quad
        % pmatrix格式,在矩阵两侧加小括号
        \begin{pmatrix}
            0 & 1 \\
            1 & 0
        \end{pmatrix} \quad
        % bmatrix格式,在矩阵两侧加方括号
        \begin{bmatrix}
            0 & 1 \\
            1 & 0
        \end{bmatrix} \quad
        % Bmatrix格式,在矩阵两侧加花括号
        \begin{Bmatrix}
            0 & 1 \\
            1 & 0
        \end{Bmatrix} \quad
        % vmatrix格式,在矩阵两侧加单个竖线
        \begin{vmatrix}
            0 & 1 \\
            1 & 0
        \end{vmatrix} \quad
        % Vmatrix格式,在矩阵两侧加双竖线
        \begin{Vmatrix}
            0 & 1 \\
            1 & 0
        \end{Vmatrix} \quad
    \]
    
    % 常用的省略号:dots, vdots, ddots
    \[
        A = \begin{bmatrix}
            a_{11} & \dots & a_{1n} \\
            \vdots & \ddots & \vdots \\
            0 & \dots & a_{nn}
        \end{bmatrix}_{n \times n}
    \]

    % 跨行的省略号 \hdotsfor{行数}
    \[
        \begin{pmatrix}
            1 & \frac 12 & \dots & \frac 1n \\
            \hdotsfor{4} \\
            m & \frac ms & \dots & frac mn
        \end{pmatrix}
    \]

    % 还有行间的小矩阵,也就是矩阵高度和行高度一样
    公式$ z = (x,y)$也可以使用
    \begin{math}
        \left(
            \begin{smallmatrix}
                x & -y \\ y & x
            \end{smallmatrix}
        \right)
    \end{math}来表示

    % array环境类似tabular,在可选参数中可以选择 r l c显示竖线
    \[
        \begin{array}{r|r}
            \frac 12 & 0\\
            \hline
            0 & -\frac abc \\
        \end{array}
    \]
\end{document}
